% document class
\documentclass[titlepage]{article}

% document layout packages 
\usepackage{geometry} % allows easier page formatting
\geometry{left=2.5cm,right=2.5cm,top=2.5cm,bottom=2.5cm}

% formatting
\usepackage{indentfirst} % suppresses inbuilt "no-indent" after section
\usepackage{amsmath} % amsmath package
\usepackage[english]{isodate} % date package

% font (inter light)
\usepackage[sfdefault,light]{inter} %% Option 'sfdefault' only if 'inter' is to be used
\usepackage[T1]{fontenc}
\usepackage{enumerate}

\begin{document} % begin document!

% title page
\title{\textbf{Module 2 Assignment}}
\author{Nathan Shea Ouedraogo}
\date{\printdateTeX{2024/03/04}}
\maketitle

% Question 1
\section{}
\setlength{\parindent}{20pt}
\begin{enumerate}[a)]
    \item 
    FTIR Microscopy ($\mu$-FTIR) combines microscopic methods with FTIR analysis. 
    This method relies on the transmission of visible and infrared radiation through 
    a sample material and can provide detailed information about the physical structures
    of small structures. It can provide chemical and physical properties of the materials
    (such as composition, thickness, structure) via IR and map it into a physical grid 
    across the sample. A major drawback to $\mu$-FTIR is in its spatial resolution
    (typical: $2.5\mu m - 25\mu m$) which makes it unsuitable for fine measurements requiring 
    resolutions at and below a few microns. Additionally, since water is absorbs strongly in 
    the IR spectra living systems are difficult to measure. Optical-Photothermal Infrared Spectroscopy
    (O-PTIR) is a method which can overcome these limitations. A tunable laser in the mid-IR is 
    pulsed and heats the sample of interest at a specific wavelength causing absorption of the sample. 
    A visible light probe measures the resultant photothermal effect, meaning the resultant spectra 
    are dependant on visible light and not infrared light. Additionally the spatial resolution is much 
    better allowing the probing of smaller structures. Like $\mu$-FTIR, O-PTIR uses IR to 
    probe molecular bonds and cause vibrational excitations however unlike $\mu$-FTIR the measurements 
    are dependent on visible light and not IR. Finally, $\mu$-FTIR can be used for hard and soft
    sample types whereas O-PTIR is better suited for semi-hard to soft material.
    
    \item
    Infrared nanospectroscopy combine nanometer scale spectroscopic methods with infrared spectroscopy. 
    Two popular methods are AFM-IR and IR s-SNOM. AFM-IR combines atomic force microscopy (AFM)
    with infrared spectroscopy. AFM allows high resolution scanning of the surfaces of 
    structures. A cantilever is run across a the surface and any defects cause the cantilever 
    to riser or fall. This causes a change in measured potential which can be used to map the 
    physical surface of a material. In AFM-IR, the sample material is irradiated with infrared 
    radiation from a tunable laser causing absorption in the sample. The resultant expansion 
    caused by the photothermal absorption into the sample is transduced by the cantilever tip. 
    This allows for high resolution topographical mapping of IR information. IR s-SNOM (scattering-
    type Scanning Near-field Optical Microscopy) uses an AFM cantilever to provide topographical 
    mapping of a surface. Unlike AFM-IR which transduces the displacement of the cantilever 
    due to photothermal expansion, IR s-SNOM detects the \emph{scattering} of IR due to 
    the cantilever tip. In AFM-IR, the sample is irradiated whereas in IR s-SNOM the cantilever is 
    irradiated and the scattering caused by the tip is related to the absorption 
    coefficient of the material. AFM-IR is best suited for soft material 
    as they have large thermal expansion coefficients whereas s-sNOM is 
    better suited for harder materials as they scatter light the most efferently.
    
    \item
    There are two main methods of implementing machine learning; supervised and unsupervised 
    methods. In supervised learning, data is labelled before being used in training. 
    This means that the algorithm has a baseline `knowledge' of what the correct output 
    should be and what relations to look for. It is more time consuming as the data must be conditioned
    before being fed to the algorithm, cannot resolve latent variables, and is generally 
    more accurate than unsupervised methods. Unsupervised methods use unlabeled data 
    meaning the algorithm has no advanced knowledge of what relationships should be present in the 
    data. This method is very powerful for mining large amounts of data with latent variables at the cost
    of accuracy and resource use. Accuracy may suffer (especially in early stages of learning) as these algorithms
    must be trained to output nonspurious relations within the data. This also results in a higher overall 
    use of resources and much greater complexity. 
\end{enumerate}

\newpage

% Question 2
\section{}
\setlength{\parindent}{20pt}
\par \emph{k}-Means is a type of unsupervised machine learning used in data analysis. 
It takes a set of data points and divides them into a number of clusters 
determined by the user. Each cluster is averaged and the outputted data is 
reduced in size and organized by correlation. Centroids are the arithmetic center 
of a given group of data points. Data points are assigned to clusters based on 
how close they are to any \emph{k} $>$ 1 clusters. The underlying assumption is that closely spaced 
data points are correlated and represent the same variable. \emph{k}Means works well for unlabeled data 
with a known number of categories. For example this method would work well for combined spectroscopic 
and visual data where absorption bands for some feature is known but the raw spectral data is unlabeled. 
Using \emph{k}means one could set the number of clusters to the number of expected features and average the data 
into clusters corresponding to these features. 

% Question 3
\section{}
\setlength{\parindent}{20pt}
    


\end{document}