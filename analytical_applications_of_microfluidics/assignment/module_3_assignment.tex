%% PREAMBLE %%

% document class
\documentclass[titlepage]{article}

% document layout packages 
\usepackage{geometry} % allows easier page formatting
\geometry{left=2.5cm,right=2.5cm,top=2.5cm,bottom=2.5cm}
\usepackage{tabularx} % table package

% math packages
\usepackage{amsmath} % amsmath package
\usepackage{amssymb} % lots of stuff ex: bbm !
\usepackage{nicefrac} % commands to make ugly fractions nicer. not always needed!
\usepackage{bm} % bold math package !

% formatting
\usepackage{indentfirst} % suppresses inbuilt "no-indent" after section
\usepackage{setspace} % Using \doublespacing in the preamble 
\usepackage{enumerate} % lists nd stuff 
\onehalfspacing % changes the text to 1.5-line spacing

% main font (inter light)
\usepackage[sfdefault,light]{inter} %% Option 'sfdefault' only if 'inter' is to be used
\usepackage[T1]{fontenc}

% math font package
\usepackage{amsfonts} % latex defaults fractions to main font, so must use \displaystyle before \frac expression

% other
\usepackage[english]{isodate} % date package
\usepackage[colorlinks=true, urlcolor=blue, linkcolor=red]{hyperref} % hyperlink package

%% DOCUMENT %%

\begin{document} % begin document!

% title page
\title{\textbf{Module 3 Assignment}}
\author{Nathan Shea Ouedraogo}
\date{\printdateTeX{2024/04/15}}
\maketitle

\section{}
\begin{center}
    \large
    \textbf{Dimensional Reduction of a Scalar Variable}
\end{center}
\normalsize

\noindent A dimensional variable $u$ may be reduced to its non-dimensional form $\tilde{u}$ by re-centering it to the origin by a scalar $u_{r}$ and dividing it by a scalar factor $u_{s}$ :
\begin{align}
    \tilde{u} &= \displaystyle\frac{u-u_{r}}{u_{s}}
\end{align}

\noindent If we assume that the variable is already centered at the origin, $u_{r}$ disappears:
\begin{align}
    \tilde{u} &= \displaystyle\frac{u}{u_{s}}
\end{align}

\noindent Rearranging to put the equation in terms of the dimensional variable: 
\begin{align}
    u_{s}\tilde{u} &= u\hspace*{5pt}\square
\end{align}

\begin{center}
    \large
    \textbf{Dimensional Reduction of a Vector Variable}
\end{center}
\normalsize

\noindent Given a vector variable 
$\vec{u}=\left<u_{1},u_{2},\cdots,u_{n-1}\right>,\hspace*{2pt}\vec{u_{n}}\hspace*{2pt}\in\hspace*{2pt}\mathbb{R}^n$ 
with \emph{n} number of dimensional elements its non-dimensional form $\vec{\tilde{u}}$ is:

\begin{align}
    \vec{\tilde{u}} &= \langle{\displaystyle\frac{u_{1}-u_{r_{1}}}{u_{s_{2}}}, \displaystyle\frac{u_{2}-u_{r_{2}}}{u_{s_{2}}}, \cdots, \displaystyle\frac{u_{n-1}-u_{r_{n-1}}}{u_{s_{n-1}}}}\rangle
\end{align}

\noindent If we assume the vector is centered at the origin, the equation becomes: 

\begin{align}
    \vec{\tilde{u}}=\langle{\displaystyle\frac{u_{1}}{u_{s_{2}}}, \displaystyle\frac{u_{2}}{u_{s_{2}}}, \cdots, \displaystyle\frac{u_{n-1}}{u_{s_{n-1}}}}\rangle
\end{align}

\noindent If all  elements of $\vec{\tilde{u}}$ have the same dimension, we can scale some by a characteristic factor $U_{0}$:
\begin{align}
    \vec{\tilde{u}} &= \langle{\displaystyle\frac{u_{1}}{U_{0}}, \displaystyle\frac{u_{2}}{U_{0}}, \cdots, \displaystyle\frac{u_{n-1}}{U_{0}}}\rangle \\
    \vec{\tilde{u}} &= \displaystyle\frac{1}{U_{0}}\langle{u_{1}, u_{2}, \cdots, u_{n-1}}\rangle
\end{align}
\noindent And finally we can rearrange the equation in terms of the dimensional vector: 

\begin{align}
    U_{0}\vec{\tilde{u}} &= \langle{u_{1}, u_{2}, \cdots, u_{n-1}}\rangle \\
    U_{0}\vec{\tilde{u}} &= \vec{u}\hspace*{5pt}\square
\end{align}

\newpage

\begin{center}
    \textbf{\emph{i) Navier-Stokes Equation for Incompressible Liquids}}
\end{center}
\begin{align}
    \rho\left(\displaystyle\frac{\partial\vec{v}}{\partial{t}}\right)+\left(\vec{v}\cdot\vec{\nabla}\right)\vec{v} &= -\vec{\nabla}p+\eta\nabla^2\vec{v}+\rho\vec{F}
\end{align}

\begin{center}
    \textbf{\emph{ii) Dimensional Reduction of Time and Pressure}}
\end{center}

Applying Equation (1):
\begin{align}
    \tilde{t}  &=  \displaystyle\frac{t-t_{r}}{t_{s}} \\
    \tilde{p}  &=  \displaystyle\frac{p-p_{r}}{p_{s}}
\end{align}

Applying Equation (2):
\begin{align}
    \tilde{t}  &=  \displaystyle\frac{t}{t_{s}} \\
    \tilde{p}  &=  \displaystyle\frac{p}{p_{s}}
\end{align}

Applying Equation (3):
\begin{align}
    t_{s}\tilde{t}  &= t\\
    p_{s}\tilde{p} &= p\hspace*{5pt}\square 
\end{align} 
\begin{center}
    \textbf{\emph{iii) Dimensional Reduction of Velocity}}
\end{center}

\noindent The dimensional velocity vector is:

\begin{align}
    \vec{v} &= \begin{bmatrix}
        v_{x} \\
        v_{y} \\
        v_{z} 
    \end{bmatrix}
\end{align}

\noindent Applying equation (4) gives the dimensionless velocity vector:
\begin{align}
    \vec{\tilde{v}} &= \begin{bmatrix}
        \displaystyle\frac{v_{x}-v_{r_{x}}}{v_{s_{x}}} \\[10pt]
        \displaystyle\frac{v_{y}-v_{r_{y}}}{v_{s_{y}}} \\[10pt]
        \displaystyle\frac{v_{y}-v_{r_{y}}}{v_{s_{y}}} 
    \end{bmatrix}
\end{align}

\newpage
\noindent Given a characteristic velocity $V_{0}$ and using equations (5) \& (9): 
\begin{align}
    \vec{\tilde{v}} &= \displaystyle\frac{1}{V_{0}}\begin{bmatrix}
        v_{x} \\
        v_{y} \\
        v_{z}     
    \end{bmatrix} \\
    {V_{0}}\vec{\tilde{v}} &= \vec{v}\hspace*{5pt}\square
\end{align}

\begin{center}
    \textbf{\emph{iv) Dimensional Reduction of the Position Vector}}
\end{center}
The position vector is: 
\begin{align}
    \vec{f} &= \begin{bmatrix}
        x \\
        y \\
        z 
    \end{bmatrix}
\end{align}
\noindent Which by equations (4)-(7) and a characteristic length of $L_{0}$ gives the relationship: 
\begin{align}
    \vec{f}=L_{0}\vec{\tilde{f}}\hspace*{5pt}
    =L_{0}\begin{bmatrix}
        \tilde{x} \\
        \tilde{y} \\
        \tilde{z}
    \end{bmatrix}
    =\begin{bmatrix}
        L_{0}\tilde{x}\\
        L_{0}\tilde{y} \\
        L_{0}\tilde{z}
    \end{bmatrix}
    = \begin{bmatrix}
        x \\
        y \\
        z 
    \end{bmatrix}
    =\vec{f} \hspace*{5pt}\square
\end{align}

\begin{center}
    \textbf{\emph{v) Dimensional Reduction of the Pressure Gradient}}
\end{center}

\noindent The dimensional pressure gradient is:
\begin{align}
    -\vec{\nabla}p &= -\begin{bmatrix}
        \displaystyle\frac{\partial{f\left(x,y,z\right)}}{\partial{x}}p_{x} \\[0.4cm]
        \displaystyle\frac{\partial{f\left(x,y,z\right)}}{\partial{y}}p_{y} \\[0.4cm]
        \displaystyle\frac{\partial{f\left(x,y,z\right)}}{\partial{z}}p_{z}
    \end{bmatrix}
\end{align}

\noindent And its dimensionless form:  
\begin{align}
    -\vec{\tilde{\nabla}}\tilde{p} &= -\begin{bmatrix}
        \displaystyle\frac{\partial{\tilde{f}\left(\tilde{x},\tilde{y},\tilde{z}\right)}}{\partial{\tilde{x}}}\tilde{p}_{x} \\[0.4cm]
        \displaystyle\frac{\partial{\tilde{f}\left(\tilde{x},\tilde{y},\tilde{z}\right)}}{\partial{\tilde{y}}}\tilde{p}_{y} \\[0.4cm]
        \displaystyle\frac{\partial{\tilde{f}\left(\tilde{x},\tilde{y},\tilde{z}\right)}}{\partial{\tilde{z}}}\tilde{p}_{z} 
    \end{bmatrix}
\end{align}

\noindent Substituting dimensionless quantities from equations (16), (22) into equation (23) gives: 
\begin{align}
    -\vec{\nabla}p &= -\begin{bmatrix}
        \displaystyle\frac{\partial{\tilde{f}\left(\tilde{x},\tilde{y},\tilde{z}\right)}}{\partial{L_{0}\tilde{x}}}p_{s}\tilde{p}_{x} \\[0.4cm]
        \displaystyle\frac{\partial{\tilde{f}\left(\tilde{x},\tilde{y},\tilde{z}\right)}}{\partial{L_{0}\tilde{y}}}p_{s}\tilde{p}_{y} \\[0.4cm]
        \displaystyle\frac{\partial{\tilde{f}\left(\tilde{x},\tilde{y},\tilde{z}\right)}}{\partial{L_{0}\tilde{z}}}p_{s}\tilde{p}_{z} 
    \end{bmatrix}
\end{align}



\noindent Since the characteristic length is a constant, it can be pulled out: 
\begin{align}
    -\vec{\nabla}p &= -\displaystyle\frac{1}{L_{0}}\begin{bmatrix}
        \displaystyle\frac{\partial{\tilde{f}\left(\tilde{x},\tilde{y},\tilde{z}\right)}}{\partial{\tilde{x}}}p_{s_{x}}\tilde{p}_{x} \\[0.4cm]
        \displaystyle\frac{\partial{\tilde{f}\left(\tilde{x},\tilde{y},\tilde{z}\right)}}{\partial{\tilde{y}}}p_{s_{y}}\tilde{p}_{y} \\[0.4cm]
        \displaystyle\frac{\partial{\tilde{f}\left(\tilde{x},\tilde{y},\tilde{z}\right)}}{\partial{\tilde{z}}}p_{s_{z}}\tilde{p}_{z} 
    \end{bmatrix}
\end{align}

\noindent And by equation (24) gives:
\begin{align}
    -\vec{\nabla}p &= -\displaystyle\frac{1}{L_{0}}\vec{\tilde{\nabla}}p_{s}\tilde{p}\hspace*{5pt}\square
\end{align}

\begin{center}
    \textbf{\emph{vi) Dimensional Reduction of the Viscosity Function}}
\end{center}

\noindent The viscosity function is given by: 
\begin{align}
    \eta\nabla^2\vec{v} &=\eta\begin{bmatrix}
        \displaystyle\frac{\partial^{\hspace*{1pt}2}{f\left(x,y,z\right)}}{\partial{x^{2}}}v_{x} \\[0.4cm]
        \displaystyle\frac{\partial^{\hspace*{1pt}2}{f\left(x,y,z\right)}}{\partial{y^{2}}}v_{y}  \\[0.4cm]
        \displaystyle\frac{\partial^{\hspace*{1pt}2}{f\left(x,y,z\right)}}{\partial{z^{2}}}v_{z}
    \end{bmatrix}
\end{align}

\noindent And when dimensionally reduced: 

\begin{align}
    {\eta}\tilde{\nabla}^2\vec{\tilde{v}} &= {\eta}\begin{bmatrix}
        \displaystyle\frac{\partial^{\hspace*{1pt}2}{\tilde{f}\left(\tilde{x},\tilde{y},\tilde{z}\right)}}{\partial{\tilde{x}^{2}}}\tilde{v_{x}} \\[0.4cm]
        \displaystyle\frac{\partial^{\hspace*{1pt}2}{\tilde{f}\left(\tilde{x},\tilde{y},\tilde{z}\right)}}{\partial{\tilde{y}^{2}}}\tilde{v_{y}}  \\[0.4cm]
        \displaystyle\frac{\partial^{\hspace*{1pt}2}{\tilde{f}\left(\tilde{x},\tilde{y},\tilde{z}\right)}}{\partial{\tilde{z}^{2}}} \tilde{v_{z}}
    \end{bmatrix}
\end{align}

\newpage
\noindent Substituting equation (22) into equation (28) with a characteristic velocity $V_{0}$ gives: 

\begin{align}
    \eta\nabla^2\vec{v} &={\eta}\begin{bmatrix}
        \displaystyle\frac{\partial^{\hspace*{1pt}2}{\tilde{f}\left(\tilde{x},\tilde{y},\tilde{z}\right)}}{\partial{\left(L_{0}\tilde{x}\right)^{2}}}{V_{0}}\vec{\tilde{v_{x}}} \\[0.45cm]
        \displaystyle\frac{\partial^{\hspace*{1pt}2}{\tilde{f}\left(\tilde{x},\tilde{y},\tilde{z}\right)}}{\partial{\left(L_{0}\tilde{y}\right)^{2}}}{V_{0}}\vec{\tilde{v_{y}}}  \\[0.45cm]
        \displaystyle\frac{\partial^{\hspace*{1pt}2}{\tilde{f}\left(\tilde{x},\tilde{y},\tilde{z}\right)}}{\partial{\left(L_{0}\tilde{z}\right)^{2}}}{V_{0}}\vec{\tilde{v_{z}}}
    \end{bmatrix} 
\end{align}

\noindent Pulling out the characteristic length and velocity:
\begin{align}
    \eta\nabla^2\vec{v} &= \displaystyle\frac{\eta{V_{0}}}{L_{0}^2}\begin{bmatrix}
        \displaystyle\frac{\partial^{\hspace*{1pt}2}{\tilde{f}\left(\tilde{x},\tilde{y},\tilde{z}\right)}}{\partial{\tilde{x}^{2}}}\vec{\tilde{v_{x}}} \\[0.45cm]
        \displaystyle\frac{\partial^{\hspace*{1pt}2}{\tilde{f}\left(\tilde{x},\tilde{y},\tilde{z}\right)}}{\partial{\tilde{y}^{2}}}\vec{\tilde{v_{y}}}  \\[0.45cm]
        \displaystyle\frac{\partial^{\hspace*{1pt}2}{\tilde{f}\left(\tilde{x},\tilde{y},\tilde{z}\right)}}{\partial{\tilde{z}^{2}}}\vec{\tilde{v_{z}}}
    \end{bmatrix}
\end{align}

\noindent And finally, substituting equation (29): 
\begin{align}
    \eta\nabla^2\vec{v} &= \displaystyle\frac{\eta{V_{0}}}{L_{0}^2}\tilde{\nabla}^2\vec{\tilde{v}}\hspace*{5pt}\square
\end{align}  \\  \\

\begin{center}
    \textbf{\emph{vii) The Reynolds Number (Re)}}
\end{center}

\noindent The Reynolds number is a scalar value which modulates the dimensionless velocity gradient: 

\begin{align}
    Re &= \rho\left(\displaystyle\frac{L_{0}V_{0}}{\eta}\right)
\end{align}

\noindent In the following section we will see how this number is applied.

\newpage
\begin{center}
    \textbf{\emph{viii) Non-Dimensional Navier-Stokes Equation}}
\end{center}

\noindent Substituting equations (15), (16), (20), (27), and (32) into equation (10) gives: 

\begin{align}
    \rho\left(
        \displaystyle\frac{\partial{V_{0}\vec{\tilde{v}}}}{\partial{t_{s}\tilde{t}}}+\vec{\tilde{v}}\left(\vec{\tilde{\nabla}}\cdot\vec{\tilde{v}}\right)
    \right) 
    &= -\displaystyle\frac{1}{L_{0}}\vec{\tilde{\nabla}}p_{s}\tilde{p} + \displaystyle\frac{\eta{V_{0}}}{L_{0}^2}\tilde{\nabla}^2\vec{\tilde{v}} + \rho\vec{F} \\
    \rho\left(
        \displaystyle\frac{V_{0}}{t_{s}} 
        \right)\left(\displaystyle\frac{\partial{\vec{\tilde{v}}}}{\partial{\tilde{t}}}\right) + \rho\left(\vec{\tilde{v}}\left(\vec{\tilde{\nabla}}\cdot\vec{\tilde{v}}\right)
    \right)
    &= -\displaystyle\frac{1}{L_{0}}\vec{\tilde{\nabla}}p_{s}\tilde{p} + \displaystyle\frac{{\eta}{V_{0}}}{L_{0}^2}\tilde{\nabla}^2\vec{\tilde{v}} + \rho\vec{F}
\end{align}

\noindent Dividing by the highest order derivative term and assuming negligible contribution from body forces: 

\begin{align}
    \left(
        \displaystyle\frac{L_{0}^2}{\eta{V_{0}}}
    \right)
    \rho\left(
        \displaystyle\frac{V_{0}}{t_{s}} 
        \right)\left(\displaystyle\frac{\partial{\vec{\tilde{v}}}}{\partial{\tilde{t}}}\right) + \vec{\tilde{v}}\left(\vec{\tilde{\nabla}}\cdot\vec{\tilde{v}}\right)
    &= \left(\displaystyle\frac{L_{0}^2}{\eta{V_{0}}}\right)\left(-\displaystyle\frac{1}{L_{0}}\vec{\tilde{\nabla}}p_{s}\tilde{p}\right) + \left(
        \displaystyle\frac{L_{0}^2}{\eta{V_{0}}}
    \right) \displaystyle\frac{{\eta}{V_{0}}}{L_{0}^2}\tilde{\nabla}^2\vec{\tilde{v}} 
\end{align}


\begin{align}
    \rho\left(\displaystyle\frac{L_{0}^2}{\eta}\right)\left(\displaystyle\frac{\partial{\vec{\tilde{v}}}}{\partial{\tilde{t}}}\right) + \rho\left(\displaystyle\frac{L_{0}V_{0}}{\eta}\right)\left(\vec{\tilde{v}}\left(\vec{\tilde{\nabla}}\cdot\vec{\tilde{v}}\right)\right) &= -\displaystyle\frac{L_{0}}{\eta{V_{0}}}\vec{\tilde{\nabla}}p_{s}\tilde{p} + \tilde{\nabla}^2\vec{\tilde{v}}
\end{align}


\noindent And finally applying equation (33) to equation (37): 

\begin{align}
    \rho\left(\displaystyle\frac{L_{0}^2}{\eta}\right)\left(\displaystyle\frac{\partial{\vec{\tilde{v}}}}{\partial{\tilde{t}}}\right) + Re\left[\vec{\tilde{v}}\left(\vec{\tilde{\nabla}}\cdot\vec{\tilde{v}}\right)\right] &= -\displaystyle\frac{L_{0}}{\eta{V_{0}}}\vec{\tilde{\nabla}}p_{s}\tilde{p} + \tilde{\nabla}^2\vec{\tilde{v}}
\end{align}

\newpage
\section{}
\indent Per equation (33) $Re$ is defined as: 

\begin{align*}
    Re &= \rho\left(\displaystyle\frac{L_{0}V_{0}}{\eta}\right)
\end{align*}
\indent  Where $L_{0}$ and $V_{0}$ are characteristic length (hydrodynamic diameter) and characteristic velocity respectively, and $\rho$ and $\eta$ are the density and dynamic viscosity of the fluid respectively If pressure and temperature are assumed to be constant and body forces are assumed to be negligible, $\rho$ and $\eta$ can be assumed to be constants. $Re$ is therefore dependent on two forces; viscosity forces dependent on $\rho$ and $\eta$ and inertial forces dependent on $L_{0}$ and $V_{0}$. \newline
\indent We can construct two simplified mental models of the physical effect of $Re$ using our intuition and own experience. We know that viscous fluids like honey have a greater tendency to `stick' and less tendency to `flow'. When honey does flow it is slow and smooth with little bubbles or turbulence. However, upon heating honey it becomes much easier to pour which can be modeled as increasing the fluid's velocity. The flow of hot honey is noticeably more chaotic with bubbles and turbulence. Another way to increase this observed chaos would be to increase the diameter of the flowing liquid; honey flowing from the tip of a squeeze bottle is much more smooth than honey flowing from an upturned jar. \newline 
\indent In the first model where fluid flow is `smooth', the system is dominated by viscosity forces ($Re < 1$) and the fluid flow is said to be laminar:
\begin{align}
    \displaystyle\frac{\rho}{\eta} > L_{0}V_{0}
\end{align}
\indent The second model with `chaotic' flow is dominated by inertial forces ($Re$>{}>1) and the flow is said to be turbulent: 
\begin{align}
    \displaystyle\frac{\rho}{\eta} << L_{0}V_{0}
\end{align}
\indent Intermediate values are neither fully turbulent nor fully laminar and have contributions by both forces. 

\newpage
\section{}
\begin{center}
    \large
    \textbf{Hagen-Poiseuille Flow between Parallel Plates}
\end{center}
\normalsize
\noindent Given a cross section perpendicular to the flow of the fluid, $\partial\hspace*{2pt}\mathcal{C}$, the flow is said to be under Hagen-Poiseuille conditions if: 
\small
\begin{center}
    \emph{\begin{enumerate}
        \item The system is translation invariant (i.e. exact position is not needed) along a specified axis
        \item pressure is unidirectional along the given traslationally invariant axis and zero on the other axis 
        \item no body forces act upon the system
        \item For a given cross-section of a pipe, $\mathcal{C}$, the velocity at any point along the boundary (wall), $\partial\hspace*{2pt}\mathcal{C}$, is zero (no-slip condition)
        \item Pressure boundary conditions hold: 
            \begin{align*}
                    p(t_{i})&=p_{0}+\Delta{p} \\
                    p(t_{f})&=p_{0}
            \end{align*}
    \end{enumerate}}
\end{center}
\vspace*{10pt}
\normalsize
\begin{center}
    \textbf{\emph{i) Navier-Stokes Equation for Incompressible Liquids in Cartesian Coordinates}}
\end{center}

\noindent The convective term of the Navier Stokes equation (eq. 10) is: 

\begin{align*}
    \left(\vec{v}\cdot\vec{\nabla}\right)\vec{v}
\end{align*}

\noindent Since pressure is unidirectional along the z-axis the equation will expand to: 

\begin{align}
    \left(\vec{v}\cdot\vec{\nabla}\right)\vec{v}&=v_{x}\left(\displaystyle\frac{\partial}{\partial{x}}v_{z}\vec{i}\right)+v_{y}\left(\displaystyle\frac{\partial}{\partial{y}}v_{z}\vec{j}\right)+v_{z}\left(\displaystyle\frac{\partial}{\partial{z}}v_{z}\vec{k}\right)
\end{align}

\noindent Velocity on the xy plane is zero:

\begin{align}
    \left(\vec{v}\cdot\vec{\nabla}\right)\vec{v}=\left(0\right)\left(\displaystyle\frac{\partial}{\partial{x}}v_{z}\vec{i}\right)+\left(0\right)\left(\displaystyle\frac{\partial}{\partial{y}}v_{z}\vec{j}\right)+v_{z}\left(\displaystyle\frac{\partial}{\partial{z}}v_{z}\vec{k}\right) 
    =v_{z}\left(\displaystyle\frac{\partial}{\partial{z}}v_{z}\vec{k}\right) 
\end{align}

\noindent Since the z-axis is our chosen translationally invariant axis, velocity is zero:
\begin{align}
    \left(\vec{v}\cdot\vec{\nabla}\right)\vec{v}=\left(0\right)v_{z}\left(\displaystyle\frac{\partial}{\partial{z}}v_{z}\vec{k}\right)=0\hspace*{5pt}\blacksquare
\end{align}

\noindent \textbf{By equation (43), a system under Hagen-Poiseuille flow have no convection.} 

\newpage
\noindent  Since convection is zero under Hagen-Poiseuille flow, pressure is the only force which will be acting on our system. Additionally, under Pouisseuille flow pressure will be unidirectional along a single axis. Picking the z-axis, we can apply equation (43) to equation (10): 

\begingroup
    \addtolength\jot{5pt}
    \begin{align}
        0&=-\vec{\nabla}p+\eta\nabla^2\vec{v_{z}}\\   
        \displaystyle\frac{\partial{p}}{\partial{x}}+\displaystyle\frac{\partial{p}}{\partial{y}}+\frac{\partial{p}}{\partial{z}}&=\eta\left(\displaystyle\frac{\partial^2{v_{z}}}{\partial{x}^2}+\displaystyle\frac{\partial^2{v_{z}}}{\partial{y}^2}+\displaystyle\frac{\partial^2{v_{z}}}{\partial{z}^2}\right) \\ 
        \displaystyle\frac{\partial{p}}{\partial{z}}&=\eta\left(\frac{\partial^2{v_{z}}}{\partial{x}^2}+\displaystyle\frac{\partial^2{v_{z}}}{\partial{y}^2}+\frac{\partial^2{v}}{\partial{z}^2}\right) \\
    \displaystyle\frac{\partial{p}}{\partial{z}}&=\eta\left(\displaystyle\frac{\partial^2{v_{z}}}{\partial{y}^2}\right)\hspace*{5pt}\blacksquare 
\end{align} \\

\begin{center}
    \textbf{\emph{ii) Derivation of Fluid Velocity}}
\end{center}

\noindent \noindent Given Hagen-Poiseuille flow, the velocity on $\partial\hspace*{2pt}\mathcal{C}$ is zero. Moving away from $\partial\hspace*{2pt}\mathcal{C}$ towards the center of the cross section $\mathcal{C}$ will increase the velocity with a maxima at the center of $\mathcal{C}$. We will define this center point as $y_{0}=0$ . Physically, this change in velocity is due to sheer stress along the pipe walls increasing drag. Since velocity is symmetrical about $y_{0}$ the change in velocity with respect to position will be zero. Therefore  $\displaystyle\frac{\partial{v_{z}}}{\partial{y}} = 0$. Applying to equation (47) gives a simple second-order ordinary differential equation:

\begingroup
    \addtolength\jot{6pt}
    \begin{align}
        \int{\eta\displaystyle\frac{\partial^2{v_{z}}}{\partial{y}^2}dy}&=\int{\displaystyle\frac{\partial{p}}{\partial{z}}dy}\\
        \eta\displaystyle\frac{\partial{v_{z}}}{\partial{y}} + C_{0}&=\displaystyle\frac{\partial{p}}{\partial{z}}y \\
        C_{0} &=  \left(\displaystyle\frac{\partial{p}}{\partial{z}}\right)\left(\displaystyle\frac{y_{0}}{\eta}\right) - \displaystyle\frac{\partial{v_{z_{0}}}}{\partial{y}} \\
        C_{0} &=  \left(\displaystyle\frac{\partial{p}}{\partial{z}}\right)\left(\displaystyle\frac{0}{\eta}\right) - 0 \\
        C_{0} &=  0 \\
        \therefore \displaystyle\frac{\partial{p}}{\partial{z}}y&=\eta\displaystyle\frac{\partial{v_{z}}}{\partial{y}} \hspace*{5pt}\square 
    \end{align}
    
    \newpage
    \noindent The no-slip boundary condition states that the velocity on $\partial\hspace*{2pt}\mathcal{C}$ is zero. Let us define the vertical position at the boundary as $y_{1}$ and the velocity at the boundary as $v_{z_{1}}=0$. Applying to equation (53) leaves a first-order ordinary differential equation:
    \begin{align}
        \int{\eta\displaystyle\frac{\partial{v_{z}}}{\partial{y}}dy}&=\int{\displaystyle\frac{\partial{p}}{\partial{z}}ydy} \\
        v_{z}&= \left(\displaystyle\frac{y^2}{2\eta}\right)\left(\displaystyle\frac{\partial{p}}{\partial{z}}\right) + C_{1} \\
        C_{1} &= v_{z_{1}}-\left(\displaystyle\frac{y_{1}^2}{2\eta}\right)\left(\displaystyle\frac{\partial{p}}{\partial{z}}\right) \\
        C_{1}  &=-\left(\displaystyle\frac{y_{1}^2}{2\eta}\right)\left(\displaystyle\frac{\partial{p}}{\partial{z}}\right)\\
        v_{z}  &= \left(\displaystyle\frac{y^2}{2\eta}\right)\left(\displaystyle\frac{\partial{p}}{\partial{z}}\right) -\left(\displaystyle\frac{y_{1}^2}{2\eta}\right)\left(\displaystyle\frac{\partial{p}}{\partial{z}}\right) \\
        \therefore  v_{z} &= -\left(\displaystyle\frac{y_{1}^2}{2\eta}\right)\left(\displaystyle\frac{\partial{p}}{\partial{z}}\right)\left(1-\displaystyle\frac{y^2}{y_{1}^2}\right)\hspace*{5pt}\blacksquare
    \end{align}
\endgroup \\

\begin{center}
    \textbf{\emph{iii) Compliance with No-Slip Conditions}}
\end{center}
If the velocity is non-zero on the boundary of $\mathcal{C}$ , then the velocity will not be symmetrical. Physically this is due to the fluid experiencing shear stress and drag across the wall which causes non-zero velocity at the wall (`slip') and non-symmetrical velocity. No-slip conditions hold that the velocity at the wall is zero. This is trivial to prove as $y^2=y_{1}^2$ on $\partial\hspace*{2pt}\mathcal{C}$. Substituting into equation  (59) gives: 

\begin{align}
    v_{z} &= -\left(\displaystyle\frac{y_{1}^2}{2\eta}\right)\left(\displaystyle\frac{\partial{p}}{\partial{z}}\right)\left(1-\displaystyle\frac{y_{1}^2}{y_{1}^2}\right) \\
    v_{z} &= -\left(\displaystyle\frac{y_{1}^2}{2\eta}\right)\left(\displaystyle\frac{\partial{p}}{\partial{z}}\right)\left(1-1\right) \\
    v_{z} &= -\left(\displaystyle\frac{y_{1}^2}{2\eta}\right)\left(\displaystyle\frac{\partial{p}}{\partial{z}}\right)\left(0\right) \\
    \therefore v_{z} &= 0 \hspace*{5pt} \blacksquare
\end{align}

\newpage
\begin{center}
    \large
    \textbf{Hagen-Poiseuille Flow in a Circular Channel} \\
\end{center}

\normalsize
\begin{center}
    \textbf{\emph{i) Navier-Stokes Equation for Incompressible Liquids in Cylindrical Coordinates}}
\end{center}

\noindent Mapping the cartesian z-component from equation (10) to cylindrical coordinates gives: 

\begin{align}
    \rho\left(\displaystyle\frac{\partial{v_{z}}}{\partial{t}}+v_{r}\displaystyle\frac{\partial{v_{z}}}{\partial{r}}+\left(\displaystyle\frac{v_{\theta}}{r}\right)\left(\displaystyle\frac{\partial{v_{z}}}{\partial{\theta}}\right)+v_{z}\displaystyle\frac{\partial{v_{z}}}{\partial{z}}\right)&=-\displaystyle\frac{\partial{p}}{\partial{z}}+\eta\nabla^2v_{z}+\rho{g_{z}}
\end{align}

\noindent Assuming no-slip conditions:
\begin{align}    
    \displaystyle\frac{\partial{p}}{\partial{z}}&=\eta\nabla^2v_{z}
\end{align}

\noindent Expanding the Laplacian: 
\begin{align}    
    \displaystyle\frac{\partial{p}}{\partial{z}}&=\eta\left[\displaystyle\frac{1}{r}\left(\displaystyle\frac{\partial}{\partial{r}}\right)\left(r\displaystyle\frac{\partial{v_{z}}}{\partial{r}}\right)+\displaystyle\frac{1}{r^2}\left(\displaystyle\frac{\partial^2{v_{z}}}{\partial{\theta^2}}\right)+\displaystyle\frac{\partial^2{v_{z}}}{\partial{z^2}}\right]
\end{align}

\noindent Reducing terms due to no slip conditions and simplifying gives us the Poiseuillian Navier-Stokes equation in cylindrical coordinates: 
\begingroup
    \addtolength\jot{6pt}
    \begin{align}
        \displaystyle\frac{\partial{p}}{\partial{z}}&=\displaystyle\frac{\eta}{r}\left[\left(\displaystyle\frac{\partial}{\partial{r}}\right)\left(r\displaystyle\frac{\partial{v_{z}}}{\partial{r}}\right)\right] \\
        \left(\displaystyle\frac{r}{\eta}\right)\left(\displaystyle\frac{\partial{p}}{\partial{z}}\right)dr&=\partial\left(r\displaystyle\frac{\partial{v_{z}}}{\partial{r}}\right)\hspace*{5pt}\square
    \end{align}
\endgroup
\\
\begin{center}
    \textbf{\emph{ii) Derivation of Velocity}}
\end{center}
\noindent Equations (47) \& (68) are analogous, so the method used to derive equation (53) may be used here as well. As with equation (47) let the boundary conditions be $r_{0}=0$, $v_{z_{0}}=0$:
\begingroup
    \addtolength\jot{6pt}
    \begin{align}
        \int{\left(\displaystyle\frac{r}{\eta}\right)\left(\displaystyle\frac{\partial{p}}{\partial{z}}\right)dr}&=\int{\partial\left(r\displaystyle\frac{\partial{v_{z}}}{\partial{r}}\right)} \\
        \left(\displaystyle\frac{r^2}{2\eta}\right)\left(\frac{\partial{p}}{\partial{z}}\right)&=r\left(\displaystyle\frac{\partial{v_{z}}}{\partial{r}}\right)+C_{0}\\
        \because C_{0}&=\left(\displaystyle\frac{r_{0}^2}{2\eta}\right)\left(\frac{\partial{p}}{\partial{z}}\right)-r_{0}\left(\displaystyle\frac{\partial{v_{z}}}{\partial{r_{0}}}\right)=0\\
        \therefore\left(\displaystyle\frac{r^2}{2\eta}\right)\left(\frac{\partial{p}}{\partial{z}}\right)&=r\left(\displaystyle\frac{\partial{v_{z}}}{\partial{r}}\right)\hspace*{5pt}\square
    \end{align}
\endgroup

\newpage

\noindent Equations (72) \& (53) are analogous, so the same method applies:
\begingroup
    \addtolength\jot{6pt}
    \begin{align}
        \int{\left(\displaystyle\frac{r^2}{2\eta}\right)\left(\frac{\partial{p}}{\partial{z}}\right)dr}&=\int{r\left(\displaystyle\frac{\partial{v_{z}}}{\partial{r}}\right)}\\
        v_{z} &= \left(\displaystyle\frac{r^2}{4\eta}\right)\left(\displaystyle\frac{\partial{p}}{\partial{z}}\right)+C_{1}\\
        C_{1} &= v_{z_{1}} - \left(\displaystyle\frac{r_{1}^2}{4\eta}\right)\left(\displaystyle\frac{\partial{p}}{\partial{z}}\right) \\
        C_{1} &= - \left(\displaystyle\frac{r_{1}^2}{4\eta}\right)\left(\displaystyle\frac{\partial{p}}{\partial{z}}\right) \\
        \therefore v_{z}\left(r,\theta\right) &= -\left(\displaystyle\frac{1}{4\eta}\right)\left(\displaystyle\frac{\partial{p}}{\partial{z}}\right)\left(r_{1}^2-r^2\right)\hspace*{5pt}\blacksquare
    \end{align}
\endgroup \\

\begin{center}
    \textbf{\emph{iii) Compliance with No-Slip Conditions}}
\end{center}

\noindent By setting the radius $\left(r^2\right)$ to the boundary, from equation we get (77):
\begin{align}
    v_{z}\left(r,\theta\right) &= -\left(\displaystyle\frac{1}{4\eta}\right)\left(\displaystyle\frac{\partial{p}}{\partial{z}}\right)\left(r_{1}^2-r_{1}^2\right)\\
    v_{z}\left(r,\theta\right) &= -\left(\displaystyle\frac{1}{4\eta}\right)\left(\displaystyle\frac{\partial{p}}{\partial{z}}\right)\left(0\right)\\
    \therefore v_{z}\left(r,\theta\right) &= 0 \hspace*{5pt}\blacksquare
\end{align} 



\noindent We can prove no-slip conditions in a circular pipe in  cartesian coordinates as well. To convert back we set $\displaystyle\frac{\partial{p}}{\partial{z}}=\displaystyle\frac{\Delta{p}}{L_{0}}$, and $r^2=a^2$. Since $\mathcal{C}$ is a circle, any point on $\partial\hspace*{2pt}\mathcal{C}$ can be used to derive the radius: 
$r=a=\sqrt{x_{boundary}^2+y_{boundary}^2}$. If we make these substitutions into equation (77) it is trivial to prove compliance with no-slip boundary conditions:
\begin{align}
    \vec{v_{z}}\left(x,y\right) &= -\left(\displaystyle\frac{\Delta{p}}{4\eta{L_{0}}}\right)\left(a^2-\left(\sqrt{x_{boundary}^2+y_{boundary}^2}\right)^2\right) \\
    \vec{v_{z}}\left(x,y\right) &= -\left(\displaystyle\frac{\Delta{p}}{4\eta{L_{0}}}\right)\left(a^2-a^2\right) \\
    \therefore \vec{v_{z}}\left(x,y\right) &= 0 \hspace*{5pt}\blacksquare
\end{align}

\newpage
\begin{center}
    \large
    \textbf{Hagen-Poiseuille Flow in a Rectangular Channel} \\
\end{center}
\noindent Unlike a channel between parallel plates with theoretically `zero' height and circular channels which are symmetrical in length and width, rectangular channels introduce asymmetry between the x and y axis. Therefore the velocity gradient will now have x and y components. As we have done previously, we will assume pressure is only applied in the z-direction and the z-axis is translationally invariant. Starting from equation (46), our velocity function becomes: 

\begin{align}
    \nabla^2\vec{v_{z}}\left(x,y\right) = \displaystyle\frac{\partial^2{v_{z}}}{\partial{x^2}}+\displaystyle\frac{\partial^2{v_{z}}}{\partial{y^2}}
\end{align}

\noindent Unfortunately there is no known analytical solution to this differential equation and the best we can do is a Fourier approximation. Per the Fourier transform, any periodic function function $f\left(u\right), \hspace*{2pt} u \in \left[-D, +D\right]$ can be represented as a series of cosines and sines:

\begin{align}
    f\left(u\right) &= \sum_{n=odd}^{\infty}b_{n}\left(u\right)\sin(\displaystyle\frac{n\pi{u}}{D}) + \sum_{n=even}^{\infty}a_{n}\left(u\right)\cos(\displaystyle\frac{n\pi{u}}{D})
\end{align}


\noindent Let the y-axis be the height of the channel with bounds $0\leq{y}\leq{h}$, let the x-axis be the width of the channel with bounds $-w_{\displaystyle\nicefrac{1}{2}}\leq{x}\leq{+w_{\displaystyle\nicefrac{1}{2}}}$ and finally let the z-axis be the direction of the flow. The sign of the width terms denotes which side of the midpoint $\left(x=0\right)$ they are on. As we did in the last two sections, this will give boundary conditions $x \in \left[-w_{\displaystyle\nicefrac{1}{2}}, +w_{\displaystyle\nicefrac{1}{2}}\right]$, $y\in\left[0,h\right]$. As with the previous examples, velocity is constant across the y-axis (the height of the channel) but variable along the x-axis (the width of the channel). Using equation (85), the general form of the solution for equation (84) will be:

\begin{align}
    \vec{v_{z}}\left(x,y\right) &= \sum_{n=odd}^{\infty}b_{n}\left(x\right)\sin(\displaystyle\frac{n\pi{x}}{h}) + \sum_{n=even}^{\infty}a_{n}\left(x\right)\cos(\displaystyle\frac{n\pi{x}}{h}) 
\end{align}

\noindent The even expansion satisfies $f\left(-t\right)=f\left(t\right)$ whereas the odd expansion satisfies $f\left(-t\right)=-f\left(t\right)$. Only the odd terms will be used as they satisfy the boundary conditions: 
\begin{align}
    \vec{v_{z}}\left(x,y\right) &= \sum_{n=odd}^{\infty}b_{n}\left(x\right)\sin(\displaystyle\frac{n\pi{x}}{h})
\end{align}

\newpage

Subbing equation (87) into equation (84) gives: 

\begin{align}
    \nabla^2\vec{v_{z}}\left(x,y\right) &= \sum_{n=odd}^{\infty}\left(\displaystyle\frac{\partial^2{v_{z}}}{\partial{x}^2}\left[b_{n}\left(x\right)\sin(\displaystyle\frac{n\pi{x}}{h})\right] + \left[\displaystyle\frac{\partial^2{v_{z}}}{\partial{y}^2} b_{n}\left(x\right)\sin(\displaystyle\frac{n\pi{x}}{h})\right]\right) \\ 
    \nabla^2\vec{v_{z}}\left(x,y\right) &= \sum_{n=odd}^{\infty}\displaystyle\frac{\partial^2{v_{z}}}{\partial{x}^2}\left[b_{n}\left(x\right)\sin(\displaystyle\frac{n\pi{x}}{h})\right] + 0 \\
    \nabla^2\vec{v_{z}}\left(x,y\right) &= \sum_{n=odd}^{\infty}\displaystyle\frac{\partial{v_{z}}}{\partial{x}}\left[b_{n}^{'}\left(x\right)\sin(\displaystyle\frac{n\pi{x}}{h})+b_{n}^{'}\left(x\right)\displaystyle\frac{n\pi{x}}{h}\cos(\displaystyle\frac{n\pi{x}}{h})\right] \\
    \nabla^2\vec{v_{z}}\left(x,y\right) &= \sum_{n=odd}^{\infty}\left[b_{n}^{''}\left(x\right)\sin(\displaystyle\frac{n\pi{x}}{h}) - b_{n}\left(x\right)\displaystyle\frac{n^{2}\pi^{2}x^{2}}{h^2}\sin(\displaystyle\frac{n\pi}{h})\right] + \sum_{n=even}^{\infty} 2\left[b^{'}\left(x\right)\displaystyle\frac{n\pi{x}}{h}\cos(\displaystyle\frac{n\pi{x}}{h})\right]
\end{align}

Since we are only concerned with the odd terms, the equation reduces to:

\begin{align}
    \nabla^2\vec{v_{z}}\left(x,y\right) &= \sum_{n=odd}^{\infty}\left[b_{n}^{''}\left(x\right) - b_{n}\left(x\right)\displaystyle\frac{n^{2}\pi^{2}x^{2}}{h^2}\right]\sin(\displaystyle\frac{n\pi{x}}{h}) \hspace*{5pt} \square
\end{align}

\newpage
\section{}
\begin{center}
    \large
    \textbf{Experimental Verification of Hagen-Poiseuille's Law} \\
\end{center} 
\begin{center}
    \textbf{\emph{i) Derivation of Hagen-Poiseuille's Law in a Circular Pipe}}
\end{center}
\noindent Flow rate (Q) is obtained by integrating the velocity by the area of our cross-section $\mathcal{C}$. We must define some terms before performing the integration.
First, let us define the bounds of integration on $\mathcal{C}$. Given a circular cross-section, the bounds will be from the the minimum radius in the center of $\mathcal{C}$, $r_{min}=0$, to the maximum radius on $\partial\hspace*{2pt}\mathcal{C}$, $r_{max}=R$. Next we will define $\displaystyle\frac{\partial{p}}{\partial{z}}$ as $\displaystyle\frac{\partial{p}}{\partial{z}}=\displaystyle\frac{\Delta{p}}{L_{0}}$. Finally, we must define our area of integration and the Jacobian (scaling factor). The area of integration, $dA$, will be $2\pi$ and the Jacobian for cylindrical coordinates is simply $J\left(r,\theta,z\right)=r$. Now we are ready to derive Q:
\begingroup
    \addtolength\jot{6pt}
    \begin{align}
        Q&= -\int_{r_{min}}^{r_{max}}v_{z}J\left(r,\theta,z\right)dA\\
        &= -\int_{r=0}^{r=R}-\left(\displaystyle\frac{1}{4\eta}\right)
        \left(\displaystyle\frac{\Delta{p}}{L_{0}}\right)\left(R^2-r^2\right){r}2\pi{dr} \\
        &= \left(\displaystyle\frac{\pi}{2\eta}\right)
        \left(\displaystyle\frac{\Delta{p}}{L_{0}}\right)\int_{r=0}^{r=R}\left(R^2r-r^3\right)dr \\
        &=\left(\displaystyle\frac{\pi}{2\eta}\right)
        \left(\displaystyle\frac{\Delta{p}}{L_{0}}\right)\left(\displaystyle\frac{R^2r^2}{2}-\displaystyle\frac{r^4}{4}\right)\Bigr|_{r=0}^{r=R} \\
        &=\left(\displaystyle\frac{\pi}{2\eta}\right)
        \left(\displaystyle\frac{\Delta{p}}{L_{0}}\right)\left(\displaystyle\frac{R^4}{4}\right) \\
        &=\left(\displaystyle\frac{R^4\pi}{8\eta}\right)
        \left(\displaystyle\frac{\Delta{p}}{L_{0}}\right)\hspace*{5pt}\hspace*{5pt}\square
    \end{align} \\
    
    \begin{center}
        \textbf{\emph{ii) Hydraulic Resistance in a Circular Pipe}}
    \end{center}
    $\Delta p$ of a system is proportional to the flow rate and how easily the fluid can pass through the medium of $\partial\hspace*{2pt}\mathcal{C}$. The ability of a fluid to pass through a medium is called `hydraulic resistance', $R_{hyd}=\displaystyle\frac{8\eta{L_{0}}}{R^4\pi}$. Starting from equation (89): 
    \begin{align}
        \displaystyle\frac{{Q}{L_{0}}}{\Delta p} &= \displaystyle\frac{R^4\pi}{8\eta} \\
        \Delta p &= \displaystyle\frac{8\eta{L_{0}}}{R^4\pi}{Q} \\
        \Delta p &= {R_{hyd}}{Q}\hspace*{5pt}\\
        R_{hyd} &= \displaystyle\frac{\Delta p}{Q}\hspace*{5pt}\blacksquare 
    \end{align}
\endgroup

\newpage
\begin{center}
    \textbf{\emph{v) Experimental Data}}
\end{center}

\begin{tabularx}{0.8\textwidth}{
    | >{\centering\arraybackslash}X
    | >{\centering\arraybackslash}X |}
    \hline
    \textbf{$\bm{\Delta p}$ (mbar)} & \textbf{$\bm{H_{2}O}$ mass (g)} \\
    \hline
    50 & 0.0348 \\
    \hline
    250 & 0.2327 \\
    \hline
    500 & 0.484 \\
    \hline
    750 & 0.775 \\
    \hline
    950 & 0.9061 \\
    \hline
\end{tabularx} 
\vspace*{0.5cm}
\newline \noindent The data above was collected by applying pressure across a microfluidic device and collecting the mass of water at the outlet for 300s. Given STAP conditions and flow rate `Q' with units of $\displaystyle\frac{m^3}{s}$, we may derive the hydraulic resistance $R_{hyd}$ which has units of $Pa\displaystyle\frac{s}{m^3}$. Starting with equation (93):  
\begingroup
    \addtolength\jot{6pt}
    \begin{align}
        R_{hyd} &= \left(\displaystyle\frac{\Delta p\times\displaystyle\frac{100Pa}{1mbar}}{\left(mass_{H_{2}O}\times{\displaystyle\frac{m^3}{1000g}}\right)(s)}\right) \\
        R_{hyd} &= \left(\Delta p\times\displaystyle\frac{100Pa}{1mbar}\right) \times \left(\displaystyle\frac{1000g\times{s}}{mass_{H_{2}O}\times{m^3}}\right) \hspace*{5pt}\square
    \end{align}
    Using values from the first row in equation (95) gives: 
    \begin{align*}
        R_{hyd} &= \left(50mbar\times\displaystyle\frac{100Pa}{1mbar}\right) \times \left   (\displaystyle\frac{1000g\times{300s}}{0.0348g\times{m^3}}\right) \\
        \bm{R_{hyd}} & \bm{\thickapprox 4.3\times{10^{10}}Pa\displaystyle\frac{s}{m^3}}
    \end{align*}
\endgroup

\noindent Performing calculations for all experimental results gives the following: 
\vspace*{0.5cm}

\begin{tabularx}{0.8\textwidth}{
    | >{\centering\arraybackslash}X
    | >{\centering\arraybackslash}X |}
    \hline
    $\bm{\Delta p}$ $\left(mbar\right)$ & $\bm{R_{hyd}}$ $\left(Pa\right)\left(s\right){m^{-3}}$ \\
    \hline
    50 & 4.31$\times{10^{10}}$ \\
    \hline
    250 & 3.22$\times{10^{10}}$ \\
    \hline
    500 & 3.10$\times{10^{10}}$ \\
    \hline
    750 & 2.90 $\times{10^{10}}$ \\
    \hline
    950 & 3.15 $\times{10^{10}}$ \\
    \hline
\end{tabularx} 

\newpage
\noindent Hydraulic resistance is analogous to electrical resistance. As such, total hydraulic resistance in series is the sum of each `resistor'and in parallel the resistance is the sum of the reciprocal of each resistor. If we were to add a tube of a specific resistance to the above device, the total resistance would increase by the tube's hydraulic resistance $R_{tubes}$. Given $R_{tubes}=1\times{10^8}\left(Pa\right)\left(s\right){m^{-3}}$, the previously calculated resistances would become: 
\vspace*{0.5cm}

\begin{tabularx}{0.8\textwidth}{
    | >{\centering\arraybackslash}X
    | >{\centering\arraybackslash}X |}
    \hline
    $\bm{\Delta p}$ $\left(mbar\right)$ & $\bm{R_{hyd}}$ $\left(Pa\right)\left(s\right){m^{-3}}$ \\
    \hline
    50 & 4.32$\times{10^{10}}$ \\
    \hline
    250 & 3.23$\times{10^{10}}$ \\
    \hline
    500 & 3.11$\times{10^{10}}$ \\
    \hline
    750 & 2.91 $\times{10^{10}}$ \\
    \hline
    950 & 3.16 $\times{10^{10}}$ \\
    \hline
\end{tabularx} 


\end{document}